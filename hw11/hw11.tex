%%%%%%%%%%%%%%
% Homework 11 %
%%%%%%%%%%%%%%

\documentclass[letter]{article}

\usepackage{lipsum}
\usepackage[pdftex]{graphicx}
\usepackage[margin=1.5in]{geometry}
\usepackage[english]{babel}
\usepackage{listings}
\usepackage{amsthm}
\usepackage{amssymb}
\usepackage{framed}
\usepackage{amsmath, mathtools}
\usepackage{physics}
\usepackage{titling}
\usepackage{pgfplots}
\usepackage{fancyhdr}
\usepackage{tikz}
\usepackage[american]{circuitikz}

\pagestyle{fancy}

\newtheorem{theorem}{Theorem}[section]

\newenvironment{menumerate}{\edef\backupindent{\the\parindent}
  \enumerate\setlength{\parindent}{\backupindent}}
  {\endenumerate}

\lstset{language=python}

\pgfplotsset{compat=1.13}

%%%%%%%%%%%%%%
%  Doc Info  %
%%%%%%%%%%%%%%
\newcommand{\hwn}{11}
\newcommand{\class}{EE 16A}

\title{\class: Homework \hwn}
\author{David J. Lee\\3031796951\\dssd1001@berkeley.edu}

\fancyhead[L]{\class}
\fancyhead[C]{Homework \hwn}
\fancyhead[R]{\thepage}
\fancyfoot{}

%%%%%%%%%%%%%%

\begin{document}
\maketitle
\thispagestyle{empty}

\begin{menumerate}
    \item \textbf{Worked With...}

    Ilya (3031806896), James Zhu (3031793129)\\
    I worked alone on Friday morning, then met up with Ilya and James to discuss on Saturday afternoon.

    \newpage
    \item \textbf{Mechanical: Linear Least Squares}
    \begin{menumerate}
        \item \emph{Find the linear model of the form $b = xa$ that best fits the data}\\
        Using the data points, we can set up a matrix in the form $A\vec{x} = \vec{b}$:
        \begin{equation*}
            \begin{bmatrix}
                1\\2\\\vdots\\8
            \end{bmatrix}
            \begin{bmatrix}
                x
            \end{bmatrix}
            =
            \begin{bmatrix}
                4.5\\3\\\vdots\\8
            \end{bmatrix}
        \end{equation*}
        We can then use least squares to figure out the estimated value of $[x]$:
        \begin{equation*}
            [x] = (A^TA)^{-1}A^T
            \begin{bmatrix}
                4.5\\3\\\vdots\\8
            \end{bmatrix}
        \end{equation*}
        Using a calculator, we see that
        \begin{equation*}
            \begin{aligned}
                A^TA &= 204\\
                A^T\vec{b} &= 210
            \end{aligned}
        \end{equation*}
        Thus we have
        \begin{equation*}
            \boxed{x = 210/204}
        \end{equation*}
        \begin{equation*}
        \begin{aligned}
            \|\vec{e}\|^2 &= \norm{\begin{bmatrix}
                b_1\\\vdots\\b_8
            \end{bmatrix} - \begin{bmatrix}
                a_1\\\vdots\\a_8
            \end{bmatrix} x}^2 = \norm{\begin{bmatrix}
                4.5-x\\3-2x\\\vdots\\8-8x
            \end{bmatrix}}^2\\
            \Aboxed{\|\vec{e}\|^2 &=24.8235}
        \end{aligned}
        \end{equation*}
        \\
        Plotting the best fit line along with the data points,
        \begin{center}
        \begin{tikzpicture}
            \begin{axis}[ xlabel=$a$,
            ylabel={$b$},
            axis lines=middle,
            xmin=0,xmax=9,ymin=0,ymax=9]
            \addplot[olive, thick,domain=0:9] {(210/204) * x};
            \addplot[only marks] table {
                x y
                1 4.5
                2 3
                3 4.5
                4 6
                5 3.5
                6 7
                7 5.5
                8 8
            };
            \end{axis}
        \end{tikzpicture}
        \end{center}
        \item \emph{...of the form $b = x_1a+x_2$}\\
        Similarly, we are solving for
        \begin{equation*}
            \begin{bmatrix}
                x_1\\x_2
            \end{bmatrix}
            = (A^TA)^{-1}A^T
            \begin{bmatrix}
                4.5\\3\\\vdots\\8
            \end{bmatrix},
            A =
            \begin{bmatrix}
                1&1\\2&1\\\vdots&\vdots\\8&1
            \end{bmatrix}
        \end{equation*}
        \begin{equation*}
        \begin{aligned}
            (A^TA)^{-1} &=
            (\begin{bmatrix}
                1&2&\dots&8\\1&1&\dots&1
            \end{bmatrix}
            \begin{bmatrix}
                1&1\\2&1\\\vdots&\vdots\\8&1
            \end{bmatrix})^{-1}
            = \begin{bmatrix}
                204&36\\36&8
            \end{bmatrix}^{-1}
            = \frac{1}{84}
            \begin{bmatrix}
                2&-9\\-9&51
            \end{bmatrix}
            \\
            A^Tb &=
            \begin{bmatrix}
                1&2&\dots&8\\1&1&\dots&1
            \end{bmatrix}
            \begin{bmatrix}
                4.5\\3\\\vdots\\8
            \end{bmatrix}
            = \begin{bmatrix}
                210\\42
            \end{bmatrix}
        \end{aligned}
        \end{equation*}
        Putting them together, we get
        \begin{equation*}
            \begin{bmatrix}
                x_1\\x_2
            \end{bmatrix}
            = (A^TA)^{-1}A^Tb
            = \frac{1}{84}
            \begin{bmatrix}
                2&-9\\-9&51
            \end{bmatrix}
            \begin{bmatrix}
                210\\42
            \end{bmatrix}
            = \boxed{\begin{bmatrix}
                1/2\\3
            \end{bmatrix}}
        \end{equation*}
        \begin{equation*}
            \|\vec{e}\|^2 = \norm{\begin{bmatrix}
                b_1\\\vdots\\b_8
            \end{bmatrix} - \begin{bmatrix}
                a_1&1\\\vdots&\vdots\\a_8&1
            \end{bmatrix} \begin{bmatrix}
                x_1\\x_2
            \end{bmatrix}}^2 = \norm{\begin{bmatrix}
                4.5-(x_1 + x_2)\\3-(2x_1 + x_2)\\\vdots\\8-(8x_1 + x_2)
            \end{bmatrix}}^2 = \boxed{10}
        \end{equation*}
        Plotting the line along with the data points,
        \begin{center}
        \begin{tikzpicture}
            \begin{axis}[ xlabel=$a$, ylabel={$b$},
            grid=both,
            grid style={line width=.1pt, draw=gray!0},
            %major grid style={line width=.2pt,draw=gray!50},
            axis lines=middle,
            %minor tick num=2,
            xmin=0,xmax=9,ymin=0,ymax=9]
            \addplot[olive, thick,domain=0:9] {(1/2) * x + 3};
            \addplot[only marks] table {
                x y
                1 4.5
                2 3
                3 4.5
                4 6
                5 3.5
                6 7
                7 5.5
                8 8
            };
            \end{axis}
        \end{tikzpicture}
        \end{center}
        Yes, this apears to be a better fit for data.\\
        \item \emph{Show that the error vector $\vec{b}-A\vec{x}$ˆ is orthogonal to the columns of $A$ by direct manipulation.}\\
        To check orthogonality, we show that $A^T(\vec{b}-A\vec{x})=0$:
        \begin{equation*}
            \begin{aligned}
                A^T(\vec{b}-A\vec{x})
                &= A^T(\vec{b}-A(A^TA)^{-1}A^T\vec{b})\\
                &= A^T(\vec{b}-A(A^{-1}A^{T^{-1}})A^T\vec{b})\\
                &= A^T(\vec{b}-\vec{b})\\
                &= A^T(0)\\
                &= 0
            \end{aligned}
        \end{equation*}
    \end{menumerate}

    \newpage
    \item \textbf{Labelling Patients}
    \begin{menumerate}
        \item \emph{Set-up}\\
        We set up the following setup in the form $A\vec{x} = \vec{b}$:
        \begin{equation*}
            \begin{bmatrix}
                age_1&weight_1&tomosin2_1&ts1_1&chn1_1\\
                \vdots&&\ddots&&\vdots\\
                age_n&weight_n&tomosin2_n&ts1_n&chn1_n
            \end{bmatrix}
            \begin{bmatrix}
                \alpha_1\\\alpha_2\\\alpha_3\\\alpha_4\\\alpha_5
            \end{bmatrix}
            = \vec{b}
        \end{equation*}
        \item \emph{see appended iPython notebook }\\
        \begin{equation*}
            \vec{\alpha} =
            \begin{bmatrix}
            -0.15646169\\
            0.09239418\\
            0.48053974\\
            -0.5847018\\
            -0.35350734
            \end{bmatrix}
        \end{equation*}
        \item \emph{see appended iPython notebook }\\
        \begin{equation*}
            \begin{bmatrix}
            1\\
            -1\\
            -1\\
            1
            \end{bmatrix}
        \end{equation*}
        It seems that the first and last mice have diabetes.
    \end{menumerate}

    \newpage
    \item \textbf{Image Analysis}
    \begin{menumerate}
        \item \emph{Circle}\\
        We start with this setup in the form $A\vec{x} = \vec{b}$:
        \begin{equation*}
            \begin{bmatrix}
                (x^2+y^2)_1&x_1&y_1\\
                \vdots&\ddots&\vdots\\
                (x^2+y^2)_n&\dots&y_n
            \end{bmatrix}
            \begin{bmatrix}
                a\\d\\e
            \end{bmatrix}
            =
            \begin{bmatrix}
                1\\\vdots\\1
            \end{bmatrix}
        \end{equation*}
        Using this matrix, we can use the least squares formula to figure out:
        \begin{equation*}
            \begin{bmatrix}
                a\\d\\e
            \end{bmatrix}
            = (A^TA)^{-1}A^T\vec{b}
        \end{equation*}
        Then we just plug this into the general form:
        \begin{equation*}
            a(x^2+y^2) + dx + ey = 1
        \end{equation*}
        \item \emph{Ellipse}\\
        We start with this setup in the form $A\vec{x} = \vec{b}$:
        \begin{equation*}
            \begin{bmatrix}
                x^2_1&xy_1&y^2_1&x_1&y_1\\
                \vdots&&\ddots&&\vdots\\
                x^2_n&&\dots&&y_n
            \end{bmatrix}
            \begin{bmatrix}
                a\\b\\c\\d\\e
            \end{bmatrix}
            =
            \begin{bmatrix}
                1\\\vdots\\1
            \end{bmatrix}
        \end{equation*}
        Similarly, we can use the least squares formula to figure out the coefficients which then we just plug into the general form:
        \begin{equation*}
            ax^2 + bxy + cy^2 + dx + ey = 1
        \end{equation*}
        \item \emph{see appended iPython notebook}
        \\The best fit circle is:
        \begin{equation*}
            \boxed{2.3972(x^2+y^2) -3.98x -0.1124y = 1}
        \end{equation*}
        and
        \begin{equation*}
            \boxed{\frac{\norm{e}}{N} = 0.1637}
        \end{equation*}
        \item \emph{see appended iPython notebook}
        \\The best fit ellipse is:
        \begin{equation*}
            \boxed{2.9510x^2 + 0.4671xy + 3.6611y^2 -5.4159x -0.5534y = 1}
        \end{equation*}
        and
        \begin{equation*}
            \boxed{\frac{\norm{e}}{N} = 0.0681}
        \end{equation*}
        The error from the Ellipse model for least squares is significantly lower than that from the Circle model.\\
        Thus the Ellipse technique is better.
    \end{menumerate}

    \newpage
    \item \textbf{GPS Receivers}
    \begin{menumerate}
        \item \emph{see iPython}
        \item \emph{see iPython}
        \item \emph{see iPython}
        \item \emph{see iPython}
        \item \emph{see iPython}
        \item \emph{see iPython}
        \item \emph{see iPython}
    \end{menumerate}

    \newpage
    \item \textbf{My Problem} \\Find the linear model of the form $b = xa$ that best fits this random data set plotted in the x-y plane: (1,4.5),(2,3),(3,4.5),(4,6),(5,3.5),(6,7),(7,5.5),(8,8).\\
    \textbf{Solution:}
    Using the data points, we can set up a matrix in the form $A\vec{x} = \vec{b}$:
        \begin{equation*}
            \begin{bmatrix}
                1\\2\\\vdots\\8
            \end{bmatrix}
            \begin{bmatrix}
                x
            \end{bmatrix}
            =
            \begin{bmatrix}
                4.5\\3\\\vdots\\8
            \end{bmatrix}
        \end{equation*}
        We can then use least squares to figure out the estimated value of $[x]$:
        \begin{equation*}
            [x] = (A^TA)^{-1}A^T
            \begin{bmatrix}
                4.5\\3\\\vdots\\8
            \end{bmatrix}
        \end{equation*}
        Using a calculator, we see that
        \begin{equation*}
            \begin{aligned}
                A^TA &= 204\\
                A^T\vec{b} &= 210
            \end{aligned}
        \end{equation*}
        Thus we have
        \begin{equation*}
            x = 210/204
        \end{equation*}
        \\Finally, we can write the equation of the best-fit line as follows:
        \begin{equation*}
            \boxed{y = \frac{35}{34}x}
        \end{equation*}


\end{menumerate}

\end{document}
