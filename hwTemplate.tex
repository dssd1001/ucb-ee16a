%%%%%%%%%%%%%%
% Homework _ %
%%%%%%%%%%%%%%

\documentclass[letter]{article}

\usepackage{lipsum}
\usepackage[pdftex]{graphicx}
\usepackage[margin=1.5in]{geometry}
\usepackage[english]{babel}
\usepackage{listings}
\usepackage{amsthm}
\usepackage{amssymb}
\usepackage{framed}
\usepackage{amsmath, mathtools}
\usepackage{titling}
\usepackage{fancyhdr}
\usepackage{tikz}
\usepackage[american]{circuitikz}

\pagestyle{fancy}

\newtheorem{theorem}{Theorem}[section]

\newenvironment{menumerate}{\edef\backupindent{\the\parindent}
  \enumerate\setlength{\parindent}{\backupindent}}
  {\endenumerate}

\lstset{language=python}

%%%%%%%%%%%%%%
%  Doc Info  %
%%%%%%%%%%%%%%
\newcommand{\hwn}{?}
\newcommand{\class}{EE 16A}

\title{\class: Homework \hwn}
\author{David J. Lee\\3031796951\\dssd1001@berkeley.edu}

\fancyhead[L]{\class}
\fancyhead[C]{Homework \hwn}
\fancyhead[R]{\thepage}
\fancyfoot{}

%%%%%%%%%%%%%%

\begin{document}
\maketitle
\thispagestyle{empty}

\begin{menumerate}
    \item \textbf{Worked With...}

    \newpage
    \item \textbf{Problem Name}
    \begin{menumerate}
        \item \emph{Part Description}
        Start Solution.
        \item
        \item
        \item
    \end{menumerate}

    \item \textbf{Python}
    \begin{lstlisting}
class Link:
    empty = ()
    def __init__(self, first, rest=empty):
        self.first = first
        self.rest = rest
    def __getitem__(self, i):
        if i == 0:
            return self.first
        else:
            return self.rest[i-1]
    def __repr__(self):
        if self.rest:
            rest_str = ', ' + repr(self.rest)
        else:
            rest_str = ''
        return 'Link({0}{1})'.format(self.first, rest_str)
    \end{lstlisting}

    \newpage
    \item \textbf{Example}
    \begin{menumerate}
        \item $P_{device} = IV = 10mA \cdot 1.2 V = 12 mW.$
          Then determining currents using super position we get that,
          \begin{equation*}
            \begin{aligned}
                I_{out} &= I_{device} + I_2 \\
                &= 10mA + \frac{V_{out}}{R_{eq}} \\
                &= 10mA + \frac{1.2V}{1 \Omega + 2 \Omega} \\
                &= 10mA - .4A\\
                P &= 0.41 V_{sc}.
            \end{aligned}
          \end{equation*}
          This gives a final efficiency of $0.0294/V_{sc}.$

        \item \emph{Solve for all node voltages using nodal analysis.}

        Redraw the circuit with an added node 3:
        \begin{center}
            \includegraphics[scale=0.2]{exampledrawing.png}
        \end{center}
        Notice that $V_2 = V_x$ since $V_2$ is connected to the ground through the $60\Omega$ resistor. Keeping this in mind, we proceed with nodal analysis on each node (1,2,3):
        \\\\For $V_1$, we have:
        \begin{equation*}
            \frac{1}{10\Omega}(V_1 - 1V) - 10V_2 + \frac{1}{20\Omega}(V_1 - 0) + \frac{1}{50\Omega}(V_1 - 10V - v_3) = 0
        \end{equation*}
        For $V_2$, we have:
        \begin{equation*}
            \frac{1}{60\Omega}(V_2 - 0) + \frac{1}{55\Omega}(V_2 - V_3) = 0
        \end{equation*}
        For $V_3$, we have:
        \begin{equation*}
            \frac{1}{50\Omega}(V_3 + 10V - V_1) + \frac{1}{55\Omega}(V_3 - V_2) = 0
        \end{equation*}
        With some rearranging into the matrix-vector form ($Av = b$, where $v = \begin{bmatrix}
                V_1\\ V_2\\ V_3
            \end{bmatrix}$) then solving through iPython(code in the back), we get
        \begin{align}
            \Aboxed{V_1 &= 10.402V} \\
            \Aboxed{V_2 &= 0.146V}
        \end{align}

        \item \emph{What is $\vec{f}$ in this circuit? write $\vec{v}$ in terms of known quantities $(\vec{f} , \vec{b}, G, F, R)$.}

        First let us compute the product $F^T\vec{i}:$
        \begin{center}
        $\begin{bmatrix}
        -1&1&0&1&0\\
        0&-1&-1&0&1\\
        1&0&1&-1&-1
        \end{bmatrix}
        \begin{bmatrix}
        i_1\\i_2\\i_3\\i_4\\i_5
        \end{bmatrix}
        =
        \begin{bmatrix}
        -i_1+i_2+i_4\\-i_2-i_3+i_5\\i_1+i_3-i_4-i_5
        \end{bmatrix}$
        \end{center}
        From this, we know the vector $\vec{f}$ has 3 elements in it. Furthermore, we know we combined the nodes 1 and 4 into a supernode. Therefore:
        \begin{align}
            \Aboxed{\vec{f} =
            \begin{bmatrix}
            0\\0\\0
            \end{bmatrix}}
        \end{align}
        Solving the Ohm’s law equation for $\vec{i}$,
        \begin{equation*}
            \vec{i} = G(F\vec{v} + \vec{b}).
        \end{equation*}
        Now plugging this expression into KCL we see
        \begin{equation*}
            F^TG(F\vec{v} + \vec{b}) = -\vec{f}
        \end{equation*}
        Does this equation have a unique solution?
        Let us first go back to the matrix $F$, row-reducing we see $F$ has a linearly dependent column.
        \begin{equation*}
            F = \begin{bmatrix} -1&0&1\\1&-1&0\\0&-1&1\\1&0&-1\\0&1&-1\end{bmatrix} \implies \begin{bmatrix} 1&0&-1\\0&1&-1\\0&0&0\\0&0&0\\0&0&0\end{bmatrix}
        \end{equation*}
        This implies that the system is underdetermined. In order to make this system determined, we can drop one column of the $F$ matrix, and ignore one potential in $\vec{v}$. We will assign this node a potential of 0. This is equivalent to grounding a node in a circuit ($\vec{v}_{gr}$).

        Call the new matrix $F_{gr}$ and the new $\vec{f}$ as $\vec{f}_{gr}$. Then $F_{gr}$ has no null space and:
        \begin{align*}
            \vec{i} = GF_{gr}\vec{v}_{gr} + G\vec{b}
        \end{align*}
        Plugging this into KCL again, we see
        \begin{align*}
            F^T_{gr}GF_{gr}\vec{v}_{gr} + F^T_{gr}G\vec{b} = -\vec{f}_{gr}
        \end{align*}
        Thus,
        \begin{align}
            \Aboxed{F^T_{gr}GF_{gr}\vec{v}_{gr} = -\vec{f}_{gr} - F^T_{gr}G\vec{b}}
        \end{align}
    \end{menumerate}

    \newpage
    \item \textbf{Circuitikz/TikzPicture Example} My designed circuit:\\
    \begin{center}
      \begin{circuitikz}
          %rectangle%
          \draw[style=dashed, fill=blue!10] (0,1) rectangle (8,-7);

          \draw (-0.5,0) node[left]{$v_{mic1}'$} to[short, o-*] ++(1.5,0) to[R=$R/a_{1left}$] ++(2,0) to[short, -*] ++(0,-1);
          \draw (-0.5,-1) node[left]{$v_{mic2}'$} to[short, o-*] ++(7/6,0) to ++(1/3,0) to[R=$R/a_{2left}$] ++(2,0) to ++(1.5,0);
          \draw (-0.5,-2) node[left]{$v_{mic3}'$} to[short, o-*] ++(5/6,0) to ++(2/3,0) to[R=$R/a_{3left}$] ++(2,0) to[short, -*] ++(0,1);

          %top op-amp%
          \draw (6,-1.5) node[op amp] (opamp){}
              (opamp.-) to[short, -*] ++(-0.25,0) to ++(0,1) to[R=$R$] ++(2.88,0) to ++(0,-1.5)
              (opamp.+) to ++(-0.25,0) node[ground] {}
              (opamp.out) to ++(0.25,0) to[short, *-o] ++(1,0) node[right] {$v_{cancel\_left}$};

          \draw (1,0) to ++(0,-4) to[R=$R/a_{1right}$] ++(2,0) to[short, -*] ++(0,-1);
          \draw (2/3,-1) to ++(0,-4) to ++(1/3,0) to[R=$R/a_{2right}$] ++(2,0) to ++(1.5,0);
          \draw (1/3,-2) to ++(0,-4) to ++(2/3,0) to[R=$R/a_{3right}$] ++(2,0) to[short, -*] ++(0,1);

          %bottom op-amp%
          \draw (6,-5.5) node[op amp] (opamp){}
              (opamp.-) to[short, -*] ++(-0.25,0) to ++(0,1) to[R=$R$] ++(2.88,0) to ++(0,-1.5)
              (opamp.+) to ++(-0.25,0) node[ground] {}
              (opamp.out) to ++(0.25,0) to[short, *-o] ++(1,0) node[right] {$v_{cancel\_right}$};
      \end{circuitikz}
    \end{center}


\end{menumerate}

\end{document}
